\documentclass{article}


\usepackage{titlesec}
\usepackage{titling}
\usepackage[margin=0.25in]{geometry}
\usepackage{hyperref}


\titleformat{\section}
{\bfseries\Large} % 1st argument: formatting
{} % numbering
{0em} % indentation between numbering and text(?)
{} % more stuff before the title but after the numbering(?not sure abt this one as well)
[{\titlerule[1pt]}] % to have a line under the thing you're formatting, in this case sections


\titleformat{\subsection}
{\bfseries\large}
{}
{0em}
{}


\titleformat{\subsubsection}[runin]
{\bfseries}
{$\bullet$}
{0.5em}
{}


\titlespacing{\subsubsection}
{0em} % left margin
{0em} % space before section(??)
{0em} % space after section(?)


\renewcommand{\maketitle}{
    \begin{center}
    {\bfseries\huge
    %\theauthor
    Ian Lulu}
    
    \vspace{0.25em}

    Berwyn, IL {$\bullet$} +1 (708) 315-9423 {$\bullet$} \href{mailto:ianpatricklulu@gmail.com}{ianpatricklulu@gmail.com} {$\bullet$} \href{https://www.linkedin.com/in/ianlulu/}{LinkedIn} {$\bullet$} \href{https://github.com/IanLulu}{GitHub} {$\bullet$} \url{ianlulu.github.io}
    {$\bullet$} US Citizen
    
    \end{center}
}


% made a command to make the in-built \romannumeral into uppercase
\newcommand{\RomanNumeralCaps}[1]
    {\MakeUppercase{\romannumeral #1}}


\begin{document}


\title{R\'esum\'e}
\author{Ian Patrick Amal Lulu}


\maketitle
\pagenumbering{gobble} % turns off page numbering (turkey lol)


\section{Education}
\subsection{University of Illinois Chicago (UIC) \hfill \textnormal{December 2022}}
\vspace{-0.5em}
\textit{\textbf{Bachelor of Science in Computer Science} \hfill GPA: 3.33/4.00}
\\ % new line character in LaTeX
\textbf{Organizations \& Awards:} SAE, ACM, Vehicle Electronics and Systems Engineering, Competitive Gaming Club, Google Developer Student Club, Dean's List
\\
\textbf{Relevant Coursework:} Object-Oriented Languages \& Environments, Intro to Data Science, Computer Graphics \RomanNumeralCaps{1}, \\ 
%\indent\indent\indent\indent\indent\indent\indent\indent
Computer Algorithms \RomanNumeralCaps{1}, Data Structures, Intro to Differential Equations, Applied Linear Algebra


\section{Skills}
\textbf{Programming Languages:} C, C++, Java, Python, C\#, JavaScript, HTML, CSS, SQL, {\LaTeX}, Scala, F\#, MATLAB, R
\\
\textbf{Software:} Visual Studio IDEs, JetBrains IDEs, Git, Linux, MATLAB, Adobe Photoshop, Adobe Illustrator, Adobe Premiere Pro, Autodesk CAD, Microsoft Office/365, Unity
\\
\textbf{Spoken Languages:} Tagalog (native) English (native), Spanish (elementary), Japanese (elementary)


\section{Experience}
\subsection{Electronic Visualization Laboratory @ UIC \hfill \textnormal{Chicago, IL}}
\vspace{-0.5em}
\textit{Undergraduate Researcher \hfill May 2022 -- June 2022}
\\
$\bullet$ Helped with organizing, documenting, and reworking the professor's urban development data visualization research project to make it open and accessible via the web.
\\
$\bullet$ Assessed technologies such as React, WebGL, WebXR, and Unity along with Python data visualization libraries such as GeoPandas, Pyrosm, etc. for porting the application to the web.
\\
$\bullet$ Planned to publish a paper on the project and submit it to the IEEE Computer Society's \href{https://www.computer.org/digital-library/magazines/cg/cfp-metaverse-virtual-worlds}{Call for Papers}.

%\vspace{-0.5em}

\subsection{Technology Solutions (IT @ UIC) \hfill \textnormal{Chicago, IL}}
\vspace{-0.5em}
\textit{Student Computer Specialist \hfill January 2020 -- December 2021}
\\
$\bullet$ Assisted network engineers with on-site support for computer networking issues in research labs, offices, classrooms, etc. throughout campus during the pandemic.
\\
$\bullet$ Utilized TeamDynamix to track, troubleshoot, and respond to help tickets regarding issues with the campus network.
\\
$\bullet$ Checked ethernet cable connections on network switches as well as client side.
\\
$\bullet$ Performed network activations and documented the locations and port numbers for smoother activations in the future.
\\
$\bullet$ Interfaced with Cisco switches through PuTTY and SSH; used Bash and Cisco IOS commands to interface with the ports.


\section{Projects}
\subsection{\href{https://github.com/IanLulu/simple-ray-tracer}{Simple Ray Tracer} (UIC Class Project)} %\hfill \textnormal{\textit{March 2022 -- May 2022}}}
\vspace{-0.5em}
$\bullet$ An interactive ray tracing program for a computer graphics class.
\\
$\bullet$ Implemented with JavaScript and WebGL and displayed on a web page using HTML, CSS, and a Python web server.
\\
$\bullet$ Loaded in a JSON file to create a scene with ray tracing effects and settings that users can adjust.

\subsection{\href{https://github.com/uic-cs418/cs418-spring22-the-wild-card}{"The Impact of COVID-19"} (UIC Group Project)}
\vspace{-0.5em}
$\bullet$ Collaborated with 3 classmates to develop an end-to-end data visualization project on COVID-19 for a data science class.
\\
$\bullet$ Synthesized relevant data and applied data cleaning, EDA, visualization, and machine learning using various Python libraries such as pandas, scikit-learn, etc.
\\
$\bullet$ Used GitHub and Discord for team collaboration.
\\
$\bullet$ Presented analysis of details, issues, improvements, and main takeaways from the project to an audience.


\section{Activities}
\textbf{Society of Automotive Engineers (SAE @ UIC)} \hfill Fall 2021 -- Spring 2022
\\
$\bullet$ Helped with maintenance of Baja car for the Blizzard Baja 2022 competition.
\\
\textbf{St. Luke and St. Bernardine Parish Youth Group}, 2018 -- Present $\bullet$ Volunteering to raise funds for the youth group.


\end{document}